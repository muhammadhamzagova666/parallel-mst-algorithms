% Language: TeX
% -----------------------------------------------------------------------------
% File: PDC.tex
% Description: This report documents the implementation and evaluation of parallel 
%              algorithms for computing the Minimum Spanning Tree (MST) of a graph 
%              using Kruskal's and Boruvka's methods. It covers background information,
%              data structures, algorithmic approaches including parallel sorting and
%              merging techniques, dataset description, experimental results, and 
%              performance analysis.
%
% Target Users: Developers, researchers, and academics interested in parallel 
%               graph algorithms and distributed computation.
%
% Code Style: Follows structured and self-explanatory commenting best practices.
% -----------------------------------------------------------------------------

\documentclass[a4paper, 12pt]{article}
\usepackage[utf8]{inputenc}                % Specify the input encoding for compatibility.
\usepackage[T1]{fontenc}                   % Use T1 font encoding for proper hyphenation.
\usepackage{graphicx}                      % Package for including images.
\usepackage{lipsum}                        % Generates filler text (for draft purposes).
\usepackage{geometry}                      % Adjusts page geometry as required.
\usepackage{titlesec}                      % Customizes section titles.
\usepackage{times}                         % Uses Times font for a professional look.
\usepackage{hyperref}                      % Enables clickable cross-references.
\usepackage{amsmath}                       % Provides advanced math symbols.
\usepackage{algorithm}                     % For formatting algorithms.
\usepackage{algpseudocode}                 % For pseudocode environments.
\usepackage[linesnumbered,ruled,vlined]{algorithm2e} % Enhanced algorithm listings.
\usepackage{listings}                      % For including source code with syntax highlighting.
\usepackage{color}                         % Enables color definitions.

\geometry{a4paper, margin=0.75in}          % Set page margins for a balanced layout.
% \geometry{left=2.5cm, right=2.5cm, top=2.5cm, bottom=2.5cm}  % Alternative margin settings.

% Define custom colors for code blocks to improve readability.
\definecolor{codegreen}{rgb}{0,0.6,0}
\definecolor{codegray}{rgb}{0.5,0.5,0.5}
\definecolor{codepurple}{rgb}{0.58,0,0.82}
\definecolor{backcolour}{rgb}{0.95,0.95,0.92}

% Define a style for listings to keep code formatting consistent.
\lstdefinestyle{mystyle}{
    backgroundcolor=\color{backcolour},   % Light background for contrast.
    commentstyle=\color{codegreen},         % Color for comments in code.
    keywordstyle=\color{magenta},           % Highlight keywords.
    numberstyle=\tiny\color{codegray},        % Formatting for line numbers.
    stringstyle=\color{codepurple},         % Color for string literals.
    basicstyle=\ttfamily\footnotesize,       % Use a teletype font.
    breakatwhitespace=false,                % Do not break at whitespace.
    breaklines=true,                        % Enable automatic line breaking.
    captionpos=b,                           % Captions at the bottom of listings.
    keepspaces=true,                        % Preserve spaces in text.
    numbers=left,                           % Display line numbers on the left.
    numbersep=5pt,                          % Space between line numbers and code.
    showspaces=false,                       % Do not visually show spaces.
    showstringspaces=false,                 % Do not show space markers in strings.
    showtabs=false,                         % Do not display tab markers.
    tabsize=2                               % Set tab width to 2 spaces.
}

\lstset{style=mystyle}  % Apply the defined listing style.

\begin{document}

% -----------------------------------------------------------------------------
% Title Page
% -----------------------------------------------------------------------------
\begin{titlepage}
    \centering
    \vspace{3.5cm}  % Vertical spacing to center the content on the page.
    \includegraphics[width=0.5\textwidth]{NU-logo.jpg}\par\vspace{1cm}  % University logo.
    \includegraphics[width=0.5\textwidth]{FAST.png}\par\vspace{1cm}     % Partner institution logo.
    \vspace{1cm}
    {\scshape\LARGE\textbf{Codes for Applications of Parallel \& Distributed Computing}\par}
    \vspace{1cm}
    {\scshape\Huge\textbf{Minimum Spanning Trees using Boruvka's \& Kruskal's Algorithms}\par}
    \vspace{1cm}
    {\scshape\Large Project Report\par}
    \vspace{1cm}
    {\scshape\Large Professor Dr. Nausheen Shoaib (BCS-5E)\par}
    \vspace{1cm}
    {\scshape\Large Parallel \& Distributed Computing (CS-3006)\par}
    \vspace{1cm}
    % Listing team members with consistent styling for clarity.
    \begin{itemize}
        \item {\scshape\Large Muhammad Talha (K21-3349)\par}\vspace{0.25cm}
        \item {\scshape\Large Muhammad Hamza (K21-4579)\par}\vspace{0.25cm}
        \item {\scshape\Large Muhammad Salar (K21-4619)\par}
    \end{itemize}
    \vfill
    \vspace{1cm}    
    {Foundation of Advancement of Science and Technology\par}
    {National University of Computer and Emerging Sciences\par}
    {Department of Computer Science\par}
    {Karachi, Pakistan\par}
    {Thursday, December 7, 2023\par}
\end{titlepage}

% -----------------------------------------------------------------------------
% Table of Contents
% -----------------------------------------------------------------------------
\begin{titlepage}
\tableofcontents  % Automatically generated table of contents for easy navigation.
\end{titlepage}

% -----------------------------------------------------------------------------
% Abstract Section
% -----------------------------------------------------------------------------
\begin{abstract}
% The abstract provides a concise summary of the project's goals, methodologies,
% and key findings. It enables readers to quickly grasp the project's scope.
In this project, we implement and compare two parallel algorithms for finding the 
minimum spanning tree (MST) of a graph: Kruskal’s algorithm and Boruvka’s algorithm. 
We parallelize these algorithms using OpenMP and MPI, allowing us to evaluate their 
scalability and performance across different process counts (2, 4, and 8) and dataset 
sizes (10, 100, 1000, 5000, and 10000 nodes). Detailed performance metrics such as 
execution time, computation time, and communication time are analyzed to assess the 
advantages and limitations of each approach.
\end{abstract}

% -----------------------------------------------------------------------------
% Background Section
% -----------------------------------------------------------------------------
\section{Background}
% This section introduces the concept of a minimum spanning tree (MST), outlines 
% the applications of MSTs, and presents an overview of the algorithms used to compute them.
A minimum spanning tree (MST) is a subset of edges in a connected, edge-weighted undirected 
graph that connects all nodes with the smallest total weight. MSTs have widespread use in network 
design, clustering, image segmentation, and many other fields. Prominent algorithms to solve the MST 
problem include Prim’s, Kruskal’s, and Boruvka’s algorithms. In this project, we focus on comparing 
parallel implementations of Kruskal’s and Boruvka’s algorithms.

% -----------------------------------------------------------------------------
% Data Structures Section
% -----------------------------------------------------------------------------
\subsection{Data Structures}
% Two main data structures underpin our implementations: the undirected graph and the disjoint-set.
% The graph stores vertices and edges, while the disjoint-set (implemented via union-find) is used
% to efficiently determine connectivity and prevent cycles in the MST during edge selection.
\subsubsection{Undirected Graph}
Both algorithms are edge-centric; hence, we represent the undirected graph as a set of edges. 
An \texttt{Edge} consists of two endpoints (\texttt{src} and \texttt{dest}) and a \texttt{weight}. 
The \texttt{Graph} structure holds the number of vertices and an array of these \texttt{Edge} structures. 
Ordering is maintained such that \texttt{src < dest} to avoid duplicate edges.

\subsubsection{Disjoint-set}
To quickly check for cycles and merge connected components, we implement a disjoint-set data 
structure using the union-find algorithm. Optimizations such as path compression and union-by-rank 
are employed to ensure near-constant time performance for \texttt{find} and \texttt{union} operations. 
The data structure is vital for both the parallel Kruskal's and Boruvka's algorithm implementations.

% -----------------------------------------------------------------------------
% Inputs and Outputs Section
% -----------------------------------------------------------------------------
\subsection{Inputs and Outputs}
% The project accepts graph inputs either from a file or generated dynamically based on user-specified 
% parameters. The input file format includes the number of vertices and edges followed by the edge definitions.
The input can be provided as a file containing graph data or generated randomly based on user input. 
The file format is:
\begin{itemize}
    \item \texttt{V E src[0] dest[0] weight[0] src[1] dest[1] weight[1] ... src[E-1] dest[E-1] weight[E-1]}
\end{itemize}
In the random generation scenario, users specify the number of vertices (\texttt{V}) and the graph density 
(\texttt{D}; a value between 0 and 1). Edges are generated accordingly, and the MST, along with performance metrics, 
is output to standard output.

% -----------------------------------------------------------------------------
% The following sections continue with details on algorithmic approach, dataset description, and results.
% Further inline comments and descriptive string literals are embedded throughout the document to facilitate 
% debugging and future modifications.
% -----------------------------------------------------------------------------

\end{document}